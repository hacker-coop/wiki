\documentclass[fontsize=11pt,pagesize,parskip=half]{scrartcl}

% encoding
\usepackage{iftex}
\ifPDFTeX % latex or pdflatex
	\usepackage[utf8]{inputenc}
	\usepackage[T1]{fontenc} % fontselection
\else % xelatex or lualatex
	\usepackage{fontspec}
\fi

\usepackage[ngerman]{babel} % lang=de

% page setup
\usepackage[left=25mm,right=2cm, top=2cm, bottom=25mm]{geometry}

% font aufhübschen
\usepackage{lmodern}
\usepackage{textcomp} % § wird schöner


\usepackage[
	clausemark=forceboth,
	%~ juratotoc,
	%~ juratocnumberwidth=2.5em
]{scrjura}

\usepackage{hyperref} % support hyperlinks

\begin{document}

\author{Gründungsversammlung}
\subject{Satzung}
\title{VEBIT eV}
\subtitle{Verein zur Erschließung neuer Betätigungsformen in der Informationstechnologie e.V.}
\date{25.\,02.\,2019}
\maketitle

%~ \tableofcontents

%~ \addsec{Präambel}

\appendix

\begin{contract}

% §1
\Clause{title={Name}}

%~ \Sentence\ignorespaces
Der Verein trägt den Namen \emph{Verein zur Erschließung Neuer Betätigungsformen in der Informationstechnologie e.V.} .
%~ \Sentence\ignorespaces
Er soll in das Vereinsregister eingetragen werden.
%~ \Sentence\ignorespaces
Sitz des Vereins ist Dresden.

% §2
\Clause{title={Zweck}}

%~ \Sentence\ignorespaces
Zweck des Vereines ist die Vernetzung von IT-Dienstleistenden und verwandter Fachpersonen mit dem Ziel der Förderung der zukünftigen Gründung einer Produktiv- und Fördergenossenschaft.
%~ \Sentence\ignorespaces
Der Verein veranstaltet hierzu Tagungen, organisiert Erfahrungsaustausch, macht Beratungsangebote, sammelt Spenden, vergibt Preise, Zuschüsse und Stipendien und ergreift darüber hinaus alle zur Verfolgung des Vereinszwecks für sinnvoll erachteten Maßnahmen.

% §3
\Clause{title={Mitgliedschaft}}


Die Mitgliedschaft im Verein steht allen natürlichen und juristischen Personen offen, die den Vereinszweck zu fördern in der Lage sind.
Sie wird erworben durch Antrag beim Vorstand, Beschluss des Vorstandes über die Aufnahme und schriftliche Mitteilung der Aufnahme durch ein Vorstandsmitglied.

Die Mitgliedschaft endet
\begin{itemize}
\item durch schriftliche Austrittserklärung gegenüber einem Vorstandsmitglied
\item bei natürlichen Personen durch deren Tod
\item bei juristischen Personen durch Löschung, Auflösung, Liquidation, Eröffnung des Insolvenzverfahrens, Fusion, Rechtsformänderung, wesensveränderndem Eigentümerwechsel oder sonstiger Beendigung der rechtlichen Existenz oder
\item durch Ausschluss aus dem Verein.
\end{itemize}

Der Vorstand kann Mitglieder wegen vereinsschädigenden Verhaltens sowie wegen Verstoßes gegen Verpflichtungen gem.~\ref{p:recht-und-pflichten} aus dem Verein ausschließen.
Der Ausschluss ist per Einschreibebrief an die letzte mitgeteilte Adresse des betreffenden Mitglieds mitzuteilen und wird im Zeitpunkt der Absendung des Briefes wirksam.
Dem Ausgeschlossenen steht die unverzügliche Berufung bei der Mitgliederversammlung offen; diese entscheidet gegebenenfalls über die rückwirkende Unwirksamkeit des Ausschlusses.

% §4
\Clause{title={Rechte und Pflichten der Mitglieder}}
\label{p:recht-und-pflichten}

Alle Vereinsmitglieder haben die Pflicht
\begin{itemize}
\item jederzeit die Interessen des Vereins zu wahren
\item festgesetzte Umlagen und Beiträge unverzüglich bei Fälligkeit zu zahlen
\item dem Vorstand laufend und unverzüglich ihre aktuelle Postanschrift mitzuteilen
\item auf gesonderte Ladung durch den Vorstand hin an Vorstandssitzungen und Mitgliederversammlungen teilzunehmen.
\end{itemize}

Alle Vereinsmitglieder haben das Recht
\begin{itemize}
\item an der Mitgliederversammlung teilzunehmen und dort abzustimmen. Juristische Personen nehmen durch ihre gesetzlichen Vertreter teil und üben durch diese ihre Mitgliedschaftsrechte aus.
\item vom Vorstand unter Nennung gewünschter Tagesordnungspunkte und gegebenenfalls abzustimmender Beschlussvorlagen die Einberufung einer Mitgliederversammlung zu verlangen, die binnen vier Wochen nach Eingang beim Vorstand stattzufinden hat.
\end{itemize}

% §5
\Clause{title={Organe}}


Organe des Vereins sind
\begin{itemize}
\item die Mitgliederversammlung,
\item der Vorstand.
\end{itemize}

Mitgliederversammlung\\
Die Mitgliederversammlung tritt mindestens einmal pro Kalenderjahr zusammen; dabei soll die erste Mitgliederversammlung des Kalenderjahres vor dem 31. März stattfinden.
Die Mitgliederversammlung wird vom Vorstand mit einer Ladungsfrist von mindestens 10 Tagen einberufen.
Es ist eine Tagesordnung anzugeben.
Die Einberufung bedarf der Schriftform; elektronische Übermittlung genügt.
Mitgliederversammlungen finden stets als Präsenzveranstaltung statt.
Jedes Mitglied hat eine Stimme.
Stimmrechtsübertragungen sind unzulässig und unwirksam.
Vertreter juristischer Personen haben ihre Vertretungsberechtigung nachzuweisen.
Die Mitgliederversammlung fasst ihre Beschlüsse mit einfacher Mehrheit der abgegebenen Stimmen.
Stimmenthaltungen werden nicht mitgezählt.
Jede ordnungsgemäß geladene Mitgliederversammlung ist beschlussfähig.
Versammlungsleiter ist der 1. Vorsitzende oder ein von der Mitgliederversammlung zu Beginn zu wählender Versammlungsleiter.
Über die Beschlüsse der Mitgliederversammlung ist eine Niederschrift anzufertigen, die allen Mitgliedern binnen vier Wochen nach der Versammlung zuzustellen ist; elektronische Übermittlung genügt.

Vorstand\\
Der Vorstand besteht aus
\begin{itemize}
\item dem 1. Vorsitzenden,
\item dem 2. Vorsitzenden.
\end{itemize}
Diese werden von der Mitgliederversammlung gewählt und sind Vorstand im Sinne des \href{https://dejure.org/gesetze/BGB/26.html}{§\,26~BGB}. Sie vertreten den Verein jeweils allein gerichtlich und außergerichtlich.\\
Darüber hinaus kann die Mitgliederversammlung bis zu drei Beisitzer als stimmberechtigte Vorstandsmitglieder wählen. Der Vorstand fasst seine Beschlüsse mit einfacher Mehrheit der abgegebenen Stimmen, Stimmenthaltungen werden nicht mitgezählt. Der Vorstand regelt seine Geschäftsverteilung durch gesonderten Vorstandsbeschluss und darf einzelnen Vereinsmitgliedern Vollmachten zur Vertretung des Vereins in bestimmten Angelegenheiten erteilen. Vorstandssitzungen werden vom 1. oder 2. Vorsitzenden mit einer Ladungfrist von drei Tagen schriftlich einberufen. Elektronische Übermittlung genügt. Der Vorstand kann einzelne Vereinsmitglieder zu seinen Beratungen hinzuziehen, diese haben dann kein Stimmrecht im Vorstand. Der Einberufende kann bestimmen, dass die Sitzung in fernmündlicher oder mittels sonstiger elektronischer Nachrichtenübermittlung durchgeführter Konferenz geschieht. Über die Beschlüsse ist eine Niederschrift anzufertigen, die allen Vorstandmitgliedern binnen einer Woche nach der Sitzung zuzustellen ist; elektronische Übermittlung genügt. Die Vorstandsmitglieder bleiben im Amt, bis die Mitgliederversammlung einen Nachfolger wählt. Dies soll in der Regel nach zweijähriger Amtzeit geschehen; die Wiederwahl ist zulässig. Erklärt ein Beisitzer gegenüber dem 1. oder 2. Vorsitzenden seinen Rücktritt, so scheidet er damit aus. Erklärt der 1. oder 2. Vorsitzende gegenüber dem 2. bzw. 1. Vorsitzenden seinen Rücktritt, so wählt die Mitgliederversammlung binnen vier Wochen einen Nachfolger.\\
Der Vorstand ist ermächtigt, gegebenenfalls durch Vorstandsbeschluss Änderungen an der Satzung vorzunehmen, von denen das Registergericht die Eintragung ins Vereinsregister abhängig macht.

% §6
\Clause{title={Finanzen}}

Der Verein finanziert seine Tätigkeit durch Beiträge, Umlagen und Spenden, ferner durch Erlöse aus Veranstaltungen und sonstigen dem Vereinszweck dienenden Maßnahmen.
Der Verein ist nicht auf eigenwirtschaftliche Tätigkeit ausgerichtet und soll keine Gewinne erzielen, sondern vielmehr als Idealverein wirken.
Entstehende Kosten bzw. Defizite sind durch Beiträge und Umlagen der Mitglieder auszugleichen.
Beitrags- und Umlageverpflichtungen werden durch eine vom Vorstand erlassene Beitragsordnung geregelt.
Über die Annahme von Spenden entscheidet der Vorstand.
Ein Vereinsmitglied wird vom Vorstand mit der Verwaltung der Vereinsfinanzen beauftragt und gesondert bevollmächtigt; dieses berichtet direkt an den Vorstand.
Der gesamte Vorstand ist gegenüber der Mitgliederversammlung für die Finanzen des Vereins verantwortlich und hat dieser mindestens einmal im Kalenderjahr einen Finanzbericht zu erstatten, worauf die Mitgliederversammlung über die Entlastung des Vorstandes abzustimmen hat.

% §7
\Clause{title={Auflösung}}

Die Auflösung des Vereins geschieht durch Beschluss der Mitgliederversammlung.
Diese hat einen Liquidator zu bestellen.
Diesbezügliche Bekanntmachungen erfolgen im Elektronischen Bundesanzeiger.
Ein evtl. Liquidationserlös wird nicht an die Vereinsmitglieder ausgeschüttet, sondern kommt einem von der Mitgliederversammlung zu bestimmenden Zweck bzw. Empfänger zu.
Mangels anderslautenden Beschlusses fällt der Liquidationserlös an das Land Sachsen.

% §8
\Clause{title={Satzungsänderungen}}

Über Änderungen und Ergänzungen dieser Satzung entscheidet die Mitgliederversammlung.
Sollten einzelne Bestimmungen dieser Satzung unwirksam sein oder werden, so gelten sie als durch solche wirksame Satzungsbestimmungen ersetzt, die dem Zweck der jeweiligen Bestimmung am nächsten kommen.

\end{contract}

\vfill
Errichtet am 25.\,Februar~2019. Eingetragen am 1.\,April~2019 am Amtsgericht Dresden unter VR~11346.

VEBIT e.V., Riesaer Straße 32, 01127 Dresden

\end{document}
